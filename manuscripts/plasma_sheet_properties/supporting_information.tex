\documentclass[draft,jgrga]{agutexSI2019}

\usepackage{graphicx}
\usepackage{amssymb,amsmath,amsthm,commath}
\usepackage[capitalize]{cleveref}
\usepackage{physics}
\usepackage{siunitx}
%
%  Uncomment the following command to allow illustrations to print
%   when using Draft:
% \setkeys{Gin}{draft=false}

% Author names in capital letters:
\authorrunninghead{VO ET AL.}

% Shorter version of title entered in capital letters:
\titlerunninghead{MAGNETOTAIL PLASMA SHEET PROPERTIES FROM MMS}

%Corresponding author mailing address and e-mail address:
\authoraddr{Corresponding author: T. Vo,
Laboratory for Atmospheric and Space Physics, University of
Colorado Boulder, 1234 Innovation Drive, Boulder, CO 80303, USA.
(Tien.Vo@lasp.colorado.edu)}

\begin{document}

\title{Supporting Information for ``Identification of plasma environments within the terrestrial magnetotail and its global structure from the Magnetospheric Multiscale Mission"}

%DOI: 10.1002/%insert paper number here%

\authors{
    T. Vo\affil{1,3},
    R. E. Ergun\affil{1,2},
    M. E. Usanova\affil{1}, and
    A. Chasapis\affil{1}
}
\affiliation{1}{
    Laboratory for Atmospheric and Space Physics,
    University of Colorado,
    Boulder, CO, USA
}
\affiliation{2}{
    Department of Astrophysical and Planetary Sciences,
    University of Colorado,
    Boulder, CO, USA
}
\affiliation{3}{
    Department of Physics,
    University of Colorado,
    Boulder, CO, USA
}

%% ------------------------------------------------------------------------ %%
%  BEGIN ARTICLE
%% ------------------------------------------------------------------------ %%

\begin{article}

\noindent\textbf{Contents of this file}
%%%Remove or add items as needed%%%
\begin{enumerate}
\item Text S1
\item Figure S1
\end{enumerate}
\noindent\textbf{Additional Supporting Information (Files uploaded separately)}
\begin{enumerate}
\item Caption for Figure S1
\item Caption for large Table S1
\end{enumerate}

\noindent\textbf{Introduction}
%Type or paste your text here. The introduction gives a brief overview of the supporting information. You should include information %about as many of the following as possible (when appropriate):
% 1. a general overview of the kind of data files;
% 2. information about when and how the data were collected or created;
% 3. a general description of processing steps used;
% 4. any known imperfections or anomalies in the data.

In the following, we discuss the dataset utilized in this study. In Text S1, we present details of the compilation process that identifies periods of magnetotail observations with simultaneous availability of the FGM, FPI, EDP, and FEEPS instruments. A technical detail pertaining to the timing and quality of EDP data is provided in Figure S1. Finally, we include as a separate file a timesheet containing the identified magnetotail periods in the data in Text S1. 

% \clearpage

\noindent\textbf{Text S1. Data compilation}

There are two main features of this dataset. First, it must comprise intervals of good data availability within the magnetotail region of interest. FGM and EDP data from all four spacecrafts are needed for the curlometer and calculations of barycentric quantities. Since FPI and FEEPS time resolution are larger than those in the FIELDS suite, barycentric calculations might drastically underestimate the particle parameters. Thus, we use only measurements from MMS1 for the particles. Second, the intervals must be continuous, meaning two consecutive data points cannot be separated more than a given time period on the order of the instrument resolution. This ensures that we can properly calculate temporal correlations and moving averages.

In fast survey mode, the hierarchy of temporal resolution, in increasing order, is $\Delta t_\text{EDP}<\Delta t_\text{FGM}<\Delta t_\text{FEEPS}<\Delta t_\text{FPI}$. FEEPS data are generally available together with FGM. To compile our dataset, we start with the ephemeris data. First, the conditions ${X<0,R\geq 12\,\si{R_E}}$ and ${Q\geq 0.75}$ are imposed to roughly determine the magnetotail seasons with excellent tetrahedral formation. Then, FPI data are searched within the FGM intervals for continuous sub-intervals with a constraint ${\Delta t\leq 2\Delta t_\text{FPI}}$ on the time resolution.

\begin{figure}
\centering
\noindent\includegraphics[width=0.8\textwidth]{edp_quality.pdf}
\caption{
Typical quality flags (QF) and bitmask values (BM) for all four MMS spacecrafts during a 5-s period.
}
\label{fig:edp_quality}
\end{figure}

Continuous intervals of EDP data within FGM--FPI sub-intervals are more technically challenging to define. The EDP experiment provides quality flags (values from 0--3; 0 being very bad data or no availability; 3 being good data) and 16-bit bitmask values to indicate the occurrence of certain cautious events (zero indicating no event). Generally, a quality flag higher than 2 indicates good data. \cref{fig:edp_quality} shows a 5-s period of the typical values of these two quantities for all four spacecrafts. In the latest major release of L2 EDP data, there are quasi-periodic $\sim 0.1$-intervals (about $5\Delta t_\text{EDP}$) of shadow from the axial double probe (ADP) booms that bring the quality flag down to 1 momentarily, among other low-bit events. These periods show up as spikes in the figure and are not synchronous across MMS spacecrafts. Thus, to acquire sufficiently long intervals for analysis, we need to bypass these shadow intervals. Ignoring the quality flag, we require that either the bitmask value is 0 or between 64 and 256 (corresponding to the 6th and 8th bit). Lower bits correspond to boom shadows, probe saturation, and bad bias settings, while higher bits correspond to periods of thruster firing, during which the EDP probes are significantly shaken. Additionally, we impose a condition on the time resolution, ${\Delta t\leq 10\Delta t_\text{EDP}}$, which is more lenient than that for defining FPI continuous sub-intervals.

The compilation process is finished after the EDP sub-intervals are defined. We discard those shorter than 1 minute, since these intervals would not be useful, for example for Fourier spectrum calculations. The 1-minute OMNI dataset \cite{King2005} is also merged with this dataset. From 2017 to the end of 2020, we identify 1919 periods of magnetotail observations from this process. Their start and stop times are tabulated in Table S1, which is uploaded separately.

\noindent\textbf{Table S1.} %Type or paste caption here.
Timesheet containing the start and stop times of magnetotail observation periods in seconds and also in UTC string. The data is uploaded separately.


%%% End of body of article:
%%%%%%%%%%%%%%%%%%%%%%%%%%%%%%%%%%%%%%%%%%%%%%%%%%%%%%%%%%%%%%%%
%
% Optional Notation section goes here
%
% Notation -- End each entry with a period.
% \begin{notation}
% Term & definition.\\
% Second term & second definition.\\
% \end{notation}
%%%%%%%%%%%%%%%%%%%%%%%%%%%%%%%%%%%%%%%%%%%%%%%%%%%%%%%%%%%%%%%%


%% ------------------------------------------------------------------------ %%
%%  REFERENCE LIST AND TEXT CITATIONS

%%%%%%%%%%%%%%%%%%%%%%%%%%%%%%%%%%%%%%%%%%%%%%%

\bibliography{ref}

%%%%%%%%%%%%%%%%%%%%%%%%%%%%%%%%%%%%%%%%%%%%%%%
% if you get an error about newblock being undefined, uncomment this line:
%\newcommand{\newblock}{}
% \bibliography{ uncomment this line and enter the name of your bibtex file here } 


%% ------------------------------------------------------------------------ %%
%
%  END ARTICLE
%
%% ------------------------------------------------------------------------ %%
\end{article}
\clearpage

% Copy/paste for multiples of each file type as needed.

% enter figures and tables below here: %%%%%%%
%
%
%
%
% EXAMPLE FIGURES
% ---------------
% If you get an error about an unknown bounding box, try specifying the width and height of the figure with the natwidth and natheight options.
% \begin{figure}
%\setfigurenum{S1} %%You can change number for each figure if you want, not required. "S" prepended automatically.
% \noindent\includegraphics[natwidth=800px,natheight=600px]{samplefigure.eps}
%\caption{caption}
%\label{epsfiguresample}
%\end{figure}
%
%
% Giving latex a width will help it to scale the figure properly. A simple trick is to use \textwidth. Try this if large figures run off the side of the page.
% \begin{figure}
% \noindent\includegraphics[width=\textwidth]{anothersample.png}
%\caption{caption}
%\label{pngfiguresample}
%\end{figure}
%
%
%\begin{figure}
%\noindent\includegraphics[width=\textwidth]{athirdsample.pdf}
%\caption{A pdf test figure}
%\label{pdffiguresample}
%\end{figure}
%
% PDFLatex does not seem to be able to process EPS figures. You may want to try the epstopdf package.
%
%
% ---------------
% EXAMPLE TABLE
%
%\begin{table}
%\settablenum{S1} %%Change number for each table
%\caption{Time of the Transition Between Phase 1 and Phase 2\tablenotemark{a}}
%\centering
%\begin{tabular}{l c}
%\hline
% Run  & Time (min)  \\
%\hline
%  $l1$  & 260   \\
%  $l2$  & 300   \\
%  $l3$  & 340   \\
%  $h1$  & 270   \\
%  $h2$  & 250   \\
%  $h3$  & 380   \\
%  $r1$  & 370   \\
%  $r2$  & 390   \\
%\hline
%\end{tabular}
%\tablenotetext{a}{Footnote text here.}
%\end{table}
% ---------------
%
% EXAMPLE LARGE TABLE (UPLOADED SEPARATELY)
%\begin{table}
%\settablenum{S1} %%Change number for each table
%\caption{Time of the Transition Between Phase 1 and Phase 2\tablenotemark{a}}
%\end{table}


\end{document}

%%%%%%%%%%%%%%%%%%%%%%%%%%%%%%%%%%%%%%%%%%%%%%%%%%%%%%%%%%%%%%%

More Information and Advice:

%% ------------------------------------------------------------------------ %%
%
%  SECTION HEADS
%
%% ------------------------------------------------------------------------ %%

% Capitalize the first letter of each word (except for
% prepositions, conjunctions, and articles that are
% three or fewer letters).

% AGU follows standard outline style; therefore, there cannot be a section 1 without
% a section 2, or a section 2.3.1 without a section 2.3.2.
% Please make sure your section numbers are balanced.
% ---------------
% Level 1 head
%
% Use the \section{} command to identify level 1 heads;
% type the appropriate head wording between the curly
% brackets, as shown below.
%
%An example:
%\section{Level 1 Head: Introduction}
%
% ---------------
% Level 2 head
%
% Use the \subsection{} command to identify level 2 heads.
%An example:
%\subsection{Level 2 Head}
%
% ---------------
% Level 3 head
%
% Use the \subsubsection{} command to identify level 3 heads
%An example:
%\subsubsection{Level 3 Head}
%
%---------------
% Level 4 head
%
% Use the \subsubsubsection{} command to identify level 3 heads
% An example:
%\subsubsubsection{Level 4 Head} An example.
%
%% ------------------------------------------------------------------------ %%
%
%  IN-TEXT LISTS
%
%% ------------------------------------------------------------------------ %%
%
% Do not use bulleted lists; enumerated lists are okay.
% \begin{enumerate}
% \item
% \item
% \item
% \end{enumerate}
%
%% ------------------------------------------------------------------------ %%
%
%  EQUATIONS
%
%% ------------------------------------------------------------------------ %%

% Single-line equations are centered.
% Equation arrays will appear left-aligned.

Math coded inside display math mode \[ ...\]
 will not be numbered, e.g.,:
 \[ x^2=y^2 + z^2\]

 Math coded inside \begin{equation} and \end{equation} will
 be automatically numbered, e.g.,:
 \begin{equation}
 x^2=y^2 + z^2
 \end{equation}

% IF YOU HAVE MULTI-LINE EQUATIONS, PLEASE
% BREAK THE EQUATIONS INTO TWO OR MORE LINES
% OF SINGLE COLUMN WIDTH (20 pc, 8.3 cm)
% using double backslashes (\\).

% To create multiline equations, use the
% \begin{eqnarray} and \end{eqnarray} environment
% as demonstrated below.
\begin{eqnarray}
  x_{1} & = & (x - x_{0}) \cos \Theta \nonumber \\
        && + (y - y_{0}) \sin \Theta  \nonumber \\
  y_{1} & = & -(x - x_{0}) \sin \Theta \nonumber \\
        && + (y - y_{0}) \cos \Theta.
\end{eqnarray}

%If you don't want an equation number, use the star form:
%\begin{eqnarray*}...\end{eqnarray*}

% Break each line at a sign of operation
% (+, -, etc.) if possible, with the sign of operation
% on the new line.

% Indent second and subsequent lines to align with
% the first character following the equal sign on the
% first line.

% Use an \hspace{} command to insert horizontal space
% into your equation if necessary. Place an appropriate
% unit of measure between the curly braces, e.g.
% \hspace{1in}; you may have to experiment to achieve
% the correct amount of space.


%% ------------------------------------------------------------------------ %%
%
%  EQUATION NUMBERING: COUNTER
%
%% ------------------------------------------------------------------------ %%

% You may change equation numbering by resetting
% the equation counter or by explicitly numbering
% an equation.

% To explicitly number an equation, type \eqnum{}
% (with the desired number between the brackets)
% after the \begin{equation} or \begin{eqnarray}
% command.  The \eqnum{} command will affect only
% the equation it appears with; LaTeX will number
% any equations appearing later in the manuscript
% according to the equation counter.
%

% If you have a multiline equation that needs only
% one equation number, use a \nonumber command in
% front of the double backslashes (\\) as shown in
% the multiline equation above.

%% ------------------------------------------------------------------------ %%
%
%  SIDEWAYS FIGURE AND TABLE EXAMPLES
%
%% ------------------------------------------------------------------------ %%
%
% For tables and figures, add \usepackage{rotating} to the paper and add the rotating.sty file to the folder.
% AGU prefers the use of {sidewaystable} over {landscapetable} as it causes fewer problems.
%
% \begin{sidewaysfigure}
% \includegraphics[width=20pc]{samplefigure.eps}
% \caption{caption here}
% \label{label_here}
% \end{sidewaysfigure}
%
%
%
% \begin{sidewaystable}
% \caption{}
% \begin{tabular}
% Table layout here.
% \end{tabular}
% \end{sidewaystable}
%
%

